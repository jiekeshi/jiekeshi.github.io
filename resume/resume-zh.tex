%%
%% Copyright (c) 2018-2019 Weitian LI <wt@liwt.net>
%% CC BY 4.0 License
%%
%% Resume / Résumé
%% A short document (1-2 pages) to sum up the job-related accomplishments
%% and experience.
%%
%% Information Checklist:
%% * Contact Information
%% * Work History / Experience
%% * Education
%% * Skills
%% * Summary & Objective (optional)
%% * Hobbies & Interests (optional)
%%
%% References:
%% * CV vs. Resume: What is the Difference? When to Use Which?
%%   https://uptowork.com/blog/cv-vs-resume-difference
%% * How to Make a Resume: A Step-by-Step Guide (+30 Examples)
%%   https://uptowork.com/blog/how-to-make-a-resume
%% * Entry-Level Resume: Sample and Complete Guide (+20 Examples)
%%   https://uptowork.com/blog/entry-level-resume-example
%%
%% 2018-04-11
%%

% Chinese version
\documentclass[zh]{resume}

% File information shown at the footer of the last page


\name{杰克}{史}
%\setposition{云计算开发工程师}
%\setlocation{深圳}

%\tagline{\getposition{} @ \getlocation}
\taglineicon{\faBinoculars}
%\keywords{\getposition, \getlocation, Python, Linux, BSD}
%\photo{7em}{photo}

\socialinfo{
  \mobile{153-0145-1585}
  \email{jiekeshi125@gmail.com}
  \emailo{jiekeshi@acm.org}\\
  %\github{liweitianux} \\
  \university{中国科学院 \textbullet 计算技术研究所}
  \degree{智能信息处理重点实验室\textbullet  2020-现在} \\
  \university{扬州大学}
  \degree{电子信息工程专业 \textbullet  2016-2020} 
  \home{江苏 \textbullet 宿迁}
  \birthday{1998-10-23}
}

\begin{document}
\makeheader

%======================================================================
% Summary & Objectives
%======================================================================
%真诚应聘贵公司的\textbf{\getposition}职位,
%期待加入贵公司,帮助实现公司目标,同时获得自身成长。
\par  % NOTE: \par is needed

%======================================================================
% Competences / Skills & Languages
%\sectionTitle{技能和语言}{\faWrench}
%======================================================================
%\begin{competences}
  %\comptence{操作系统}{%
    %\icon{\faLinux} Linux (10 年);
    %\icon{\faFreebsd} BSD (DragonFly BSD 和 FreeBSD, 7 年)
  %}
  %\comptence{编程}{%
    %Python, C, Shell, R, Tcl/Tk
  %}
  %\comptence{工具}{%
    %SSH, Git, Make, Tmux, Vi, Ansible
  %}
  %\comptence{数据分析}{%
    %R, Pandas; Matplotlib, ggplot2; Keras, Scikit-learn
  %}
  %\comptence{网站开发}{%
   % Flask, JavaScript, jQuery, Bootstrap
  %}
  %\comptence{\icon{\faLanguage} 语言}{
    %\textbf{英语} --- 读写(优良), 听说(日常交流)
  %}
%\end{competences}

%======================================================================
% Education
\sectionTitle{教育背景及经历}{\faGraduationCap}
%======================================================================
\begin{educations}
\expedu
{2020.06}%
    {中国科学院}%
    {计算技术研究所}%
    {智能信息处理重点实验室}%
    {研究实习生}%
    {导师:王石博士}%
    { }

    
  \education%
    {2016.09}%
    {扬州大学}%
    {信息工程学院}%
    {电子信息工程专业本科生}%
    {毕业课题《机器人救援仿真中的多智能体行为决策研究与设计》}%
    {导师:朱俊武教授}%
    {平均成绩:83/100, Top 10\%}
\end{educations}

\sectionTitle{研究项目}{\faAtom}
%======================================================================
\begin{itemize}
\item 参与\enquote{基于新闻报道场景的 AI 辅助写稿机器人系统研发}子课题,设计事件抽取和内容生成的协同训练模型,提高data-to-text的新闻内容生成准确性
%\enquote{语义网下多智能体自动协商与博弈机制研究(201811117029Z)}
  \item 主持研究课题:
    \enquote{语义网下多智能体自动协商与博弈机制研究(201811117029Z)}(江苏省大学生创新训练计划重点项目),针对救援场景下的机器人任务分配问题,发表多篇论文,RCRSS平台测试效果远超官方样例
  \item 参与研究课题:
    \enquote{基于博弈论和机器学习的 机会型犯罪应对策略算法研究(201711117017Z)}(江苏省大学生创新训练计划重点项目),针对基于安全博弈的警力资源分配问题,合作发表一篇论文
  %\item 发表 2 篇软件著作权
\end{itemize}


%======================================================================
% Papers / Publications
\sectionTitle{发表论文}{\faFile}
%======================================================================
\begin{itemize}
  \small
  \item \textbf{Jieke Shi}, Junwu Zhu, Jian Li, Fang Liu \& Yunbo Lv\par
   \textit{An Efficient Double Auction Mechanism for Job Allocation}\par
    \textbf{Accepted} by CSCWD 2019: IEEE 23rd International Conference on Computer Supported Cooperative Work in Design 
    (CCF C)
  \item \textbf{Jieke Shi}, Zhou Yang \& Junwu Zhu\par
    \textit{A Task Allocation Algorithm in Multi-agent System via Auction}\par
    \textbf{Accepted}  ROSENET 2018: 2nd EAI International Conference on Robotic Sensor Networks
 \item \textbf{Jieke Shi}, Junwu Zhu, Zhou Yang \& Bin Li\par
    \textit{A Real-time Decentralized Algorithm for Task Scheduling in Multi-agent System with Continuous Damage}\par
    \textbf{Published} in Applied Soft Computing (SCI IF 3.907)
 \item \textbf{Jieke Shi}, Zhou Yang \& Junwu Zhu\par
    \textit{An Auction-based Rescue Task Scheduling Approach for Heterogeneous Multi-robot System}\par
    \textbf{Published} in Multimedia Tools and Applications (SCI IF 1.541, CCF C)
 \item \textbf{Jieke Shi}, Zhou Yang \& Junwu Zhu\par
    \textit{Task Scheduling: A Decentralized Negotiation Method via GGP}\par
    \textbf{Accepted} by ISAIR 2019: IEEE 4th International Symposium on Artificial Intelligence and Robotics
 \item Yi Jiang, Jinjin Wang, \textbf{Jieke Shi}, Junwu Zhu \& Ling Teng\par
    \textit{Network-aware virtual machine migration based on gene aggregation genetic algorithm}\par
    \textbf{Accepted} by Mobile Networks \& Applications (SCI IF 2.390, CCF C)
 \item Junwu Zhu, Ling Teng, Bin Li, \textbf{Jieke Shi} \& Huimin Lu\par
   \textit{Ontology Negotiation: Knowledge Interchange between Distributed Ontologies through Agent Negotiation}\par
    \textbf{Accepted} by Concurrency and Computation: Practice and Experience (SCI IF 1.167, CCF C)
 \item \textbf{Jieke Shi}, Yi Jiang, Junwu Zhu \& Xiangrui Kong\par
    \textit{A Fictitious Play Approach to Model Dynamic Personality in Swarm Agent System}\par
    \textbf{Accepted} by ISAIR 2020: IEEE 5th International Symposium on Artificial Intelligence and Robotics
\end{itemize}


% Computer Skills
\sectionTitle{相关技能}{\faCode}
%======================================================================
\begin{tabular}{llll}
\hspace{3.7em}\textbf{编程语言:}& Python, Java, C, Matlab.\\
\hspace{5.675em}工具:& SSH, Git, Tmux, Pandas, Matplotlib.\\
\hspace{3.7em}\textbf{理论知识:} & 博弈论、机器学习、强化学习等&\\
\hspace{3.7em}\textbf{英语语言:} & \lanskill{阅读}{5}  \  4级:582\\
 & \lanskill{听力}{4} \  6级:538\\
& \lanskill{口语}{3} \  雅思:6.0\\
\end{tabular}



%======================================================================
%\sectionTitle{语言水平}{\faLanguage}
%======================================================================
%\begin{tabular}{lrll}
%\hspace{3.6em}\textbf{英语:}& \lanskill{阅读}{5} & 4级:582\\
% & \lanskill{听力}{4} & 6级:538\\
%& \lanskill{口语}{3} & 雅思:6.0\\
%\end{tabular}

%======================================================================
% Awards / Scholarships / Certificates
\sectionTitle{奖励与荣誉}{\faAward}
%======================================================================
\begin{entries}
\entry{2020}%
    {\begin{itemize}
      \item 扬州大学优秀毕业生
    \end{itemize}}
      \separator{0.3em}
  \entry{2019}%
    {\begin{itemize}
    \item 第14届恩智浦杯全国大学生智能汽车竞赛华东赛区三等奖
      \item 2019年美国大学生数学建模竞赛S奖
      \item 第10届蓝桥杯Java软件开发组二等奖
      \item 国家励志奖学金
    \end{itemize}}
      \separator{0.3em}
  \entry{2018}%
   {\begin{itemize}
      \item 第8届Mathorcup全国大学生数学建模竞赛二等奖
      \item 第13届恩智浦杯全国大学生智能汽车竞赛华东赛区二等奖
      \item 第10届扬州大学挑战杯课外科技作品竞赛一等奖
      \item 扬州大学校长一等奖学金(综合成绩第一)
    \end{itemize}}
      \separator{0.3em}
  \entry{2017}%
    {{\begin{itemize}
    \item  扬州大学校长一等奖学金(综合成绩第一)
      \item 扬州大学三好学生、优秀学生干部、微科创优秀学生导师等称号
    \end{itemize}}}
\end{entries}

%======================================================================
% Internships
\sectionTitle{学生经历}{\faBriefcase}
%======================================================================
\begin{entries}
  \entry{2019}%
    {\begin{itemize}
    \item 参加复旦大学CSCW\&社会计算暑期学校,并获得优秀营员
    \item 参加CCF人工智能大会
      \item 参加CSCWD协同计算会议,并做论文报告
    \end{itemize}}
      \separator{0.3em}
  \entry{2018}%
   {\begin{itemize}
      \item 前往日本\textbullet 北九州工业大学参加ROSENET会议,做论文报告,并参观大学实验室
      \item 担任扬州大学微科创竞赛的学生指导,帮助低年级学生研发制作电子设计作品
      \item 担任扬州大学新闻社主席,承担扬州大学校报的编辑工作
    \end{itemize}}
\end{entries}

\end{document}

%% EOF
